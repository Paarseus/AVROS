\documentclass[11pt]{article}

\usepackage[margin=1in]{geometry}
\usepackage{listings}
\usepackage{xcolor}
\usepackage{hyperref}
\usepackage{enumitem}
\usepackage{booktabs}

\hypersetup{
    colorlinks=true,
    linkcolor=blue,
    urlcolor=blue
}

\lstset{
    backgroundcolor=\color{gray!10},
    basicstyle=\ttfamily\small,
    breaklines=true,
    frame=single,
    rulecolor=\color{gray!40},
    xleftmargin=6pt,
    xrightmargin=6pt,
    aboveskip=10pt,
    belowskip=10pt,
    columns=fullflexible,
    keepspaces=true
}

\lstdefinestyle{yaml}{
    basicstyle=\ttfamily\small,
    commentstyle=\color{gray},
    morecomment=[l]{\#},
}

\lstdefinestyle{error}{
    basicstyle=\ttfamily\small\color{red!70!black},
    backgroundcolor=\color{red!5},
    rulecolor=\color{red!40},
}

\title{Nav2 Testing Troubleshooting\\[4pt]
\large AVROS Ackermann Vehicle~|~Navigation Testing~|~March 2026}
\author{AVROS Team}
\date{March 2026}

\begin{document}
\maketitle
\tableofcontents
\newpage

\section{Introduction}

This document covers problems encountered during live navigation testing of the AVROS platform, following the initial bringup documented in \texttt{nav2\_bringup\_troubleshooting.tex}. All issues were discovered during on-vehicle testing with the full sensor suite (Velodyne VLP-16, Xsens MTi-680G, Intel RealSense D455) and the Nav2 navigation stack.

%% ============================================================
\section{Xsens OdometryPublisher TF Conflict}
\label{sec:xsens-odom}

\subsection{Symptom}

After launching the full navigation stack, the TF tree was split into two disconnected trees. The \texttt{map} frame and \texttt{base\_link} frame could not be connected:

\begin{lstlisting}[style=error]
[tf2_echo] Could not find a connection between 'map' and
'base_link' because they are not part of the same tree.
Tf has two or more unconnected trees.
\end{lstlisting}

Inspecting \texttt{/tf} revealed the Xsens driver was publishing a dynamic transform \texttt{imu\_link $\to$ base\_link} at 100\,Hz, which conflicted with the URDF's static \texttt{base\_link $\to$ imu\_link} transform.

\subsection{Root Cause}

The \texttt{xsens\_mti\_ros2\_driver} has three components that can publish to \texttt{/tf}:

\begin{table}[h]
\centering
\small
\begin{tabular}{@{}llll@{}}
    \toprule
    Publisher & Parameter & Publishes to /tf & Default \\
    \midrule
    TransformPublisher & \texttt{pub\_transform} & \texttt{world $\to$ imu\_link} & \texttt{true} \\
    OdometryPublisher & \texttt{pub\_odometry} & \texttt{imu\_link $\to$ base\_link} & \texttt{true} \\
    GNSSPOSEPublisher & \texttt{pub\_gnsspose} & GNSS-derived pose & \texttt{true} \\
    \bottomrule
\end{tabular}
\end{table}

The AVROS \texttt{xsens.yaml} had \texttt{pub\_transform: false} and \texttt{pub\_gnsspose: false}, but \textbf{omitted \texttt{pub\_odometry}}. Since the C++ source declares all \texttt{pub\_*} parameters with a default of \texttt{true}, the OdometryPublisher was active and publishing \texttt{imu\_link $\to$ base\_link} at 100\,Hz.

This gave \texttt{base\_link} two parents:
\begin{enumerate}[nosep]
    \item \texttt{odom $\to$ base\_link} from EKF \#1 (correct, per REP~105)
    \item \texttt{imu\_link $\to$ base\_link} from the Xsens OdometryPublisher (incorrect)
\end{enumerate}

The second parent prevented the global EKF's \texttt{map $\to$ odom} transform from connecting to the tree, because TF does not allow a frame to have multiple parents.

This is a known REP~105 violation in the Xsens driver (\href{https://github.com/xsenssupport/Xsens_MTi_ROS_Driver_and_Ntrip_Client/issues/25}{GitHub issue \#25}).

\subsection{Fix}

Add \texttt{pub\_odometry: false} to \texttt{xsens.yaml}:

\begin{lstlisting}[style=yaml]
xsens_mti_node:
  ros__parameters:
    pub_transform: false    # URDF provides base_link -> imu_link
    pub_gnsspose: false     # EKF + navsat_transform handle map frame
    pub_odometry: false     # Publishes imu_link->base_link TF (REP 105 violation)
\end{lstlisting}

\subsection{Standard Approach}

When using the Xsens driver with \texttt{robot\_localization} and Nav2, the driver should function as a \textbf{data source only}---never as a TF publisher. TF ownership should be:

\begin{itemize}[nosep]
    \item \texttt{robot\_state\_publisher}: \texttt{base\_link $\to$ imu\_link} (URDF, static)
    \item EKF \#1: \texttt{odom $\to$ base\_link} (continuous)
    \item EKF \#2: \texttt{map $\to$ odom} (continuous)
    \item Xsens driver: publishes \texttt{/imu/data}, \texttt{/gnss}, \texttt{/nmea} only
\end{itemize}

\subsection{References}

\begin{itemize}[nosep]
    \item \href{https://github.com/xsenssupport/Xsens_MTi_ROS_Driver_and_Ntrip_Client/issues/25}{GitHub \#25: Clarify implementation of Odometry in the sense of REP 105}
    \item \href{https://www.ros.org/reps/rep-0105.html}{REP 105 --- Coordinate Frames for Mobile Platforms}
\end{itemize}

%% ============================================================
\section{Global Costmap Dropping LiDAR Messages}
\label{sec:costmap-drop}

\subsection{Symptom}

The global costmap continuously dropped all incoming Velodyne point cloud messages:

\begin{lstlisting}[style=error]
[planner_server] [global_costmap]: Message Filter dropping message:
  frame 'velodyne' at time 1772424158.243 for reason 'the timestamp
  on the message is earlier than all the data in the transform cache'
\end{lstlisting}

This resulted in an empty global costmap with no obstacle data, causing the planner to either fail or produce paths through obstacles.

\subsection{Root Cause}

This was a \textbf{downstream effect} of the Xsens OdometryPublisher TF conflict (Section~\ref{sec:xsens-odom}). Because the TF tree was split into disconnected subtrees, the costmap's message filter could not look up the transform from \texttt{velodyne} to \texttt{map} (the global costmap's fixed frame). Without a valid TF chain, the message filter drops all incoming sensor messages with a timestamp mismatch error.

\subsection{Fix}

Resolving the TF tree conflict (Section~\ref{sec:xsens-odom}) fixed this issue. Once \texttt{map $\to$ odom $\to$ base\_link $\to$ velodyne} formed a complete chain, the costmap received and processed all LiDAR data normally.

\subsection{Lesson}

``Timestamp earlier than all data in the transform cache'' errors in costmap message filters almost always indicate a broken TF tree, not an actual timestamp problem. Always verify the TF tree connectivity first before investigating clock synchronization.

%% ============================================================
\section{Navigation Goal Aborted --- No Path Computed}
\label{sec:goal-abort-nopath}

\subsection{Symptom}

A navigation goal was accepted by the BT navigator but immediately aborted with zero distance remaining:

\begin{lstlisting}[style=error]
[bt_navigator] [navigate_to_pose] [ActionServer] Aborting handle.
[bt_navigator] Goal failed
\end{lstlisting}

Feedback showed \texttt{distance\_remaining: 0.0}, indicating the planner never computed a path.

\subsection{Root Cause}

This was another downstream effect of the disconnected TF tree. Without a valid \texttt{map $\to$ base\_link} transform:
\begin{enumerate}[nosep]
    \item The planner could not determine the robot's position in the \texttt{map} frame
    \item The controller could not transform the goal into the robot's local frame
    \item The BT's \texttt{ComputePathToPose} action failed, causing the BT to abort
\end{enumerate}

\subsection{Fix}

Resolved by fixing the TF tree (Section~\ref{sec:xsens-odom}). After the fix, the planner successfully computed paths (verified with direct \texttt{/compute\_path\_to\_pose} action calls returning 49-waypoint Dubin curves in 30\,ms).

%% ============================================================
\section{Navigation Goal Aborted --- Collision Detected}
\label{sec:collision}

\subsection{Symptom}

With the TF tree fixed, a navigation goal to a point 5\,m ahead was accepted, the planner computed a path, and the controller began tracking it. However, after approximately 5 seconds the controller aborted:

\begin{lstlisting}[style=error]
[controller_server] RegulatedPurePursuitController detected
  collision ahead!
[controller_server] [follow_path] [ActionServer] Aborting handle.
\end{lstlisting}

The feedback showed \texttt{distance\_remaining: 0.29\,m} --- the goal was nearly reached in the planner's estimate, but the controller detected an obstacle in the local costmap along the remaining path.

\subsection{Root Cause}

The Regulated Pure Pursuit controller has built-in collision detection (\texttt{use\_collision\_detection: true}). It projects the robot's trajectory forward along the path and checks the local costmap for obstacles within the lookahead distance. If an obstacle is detected within the time-to-collision threshold, the controller aborts rather than driving into the obstacle.

In this case, the vehicle was parked with real objects in front of it. The LiDAR-fed local costmap correctly identified these obstacles, and the controller correctly refused to execute the path.

\subsection{Analysis}

This is \textbf{correct behavior}, not a bug. The collision detection parameters are:

\begin{lstlisting}[style=yaml]
FollowPath:
  plugin: "nav2_regulated_pure_pursuit_controller::..."
  use_collision_detection: true
  max_allowed_time_to_collision_up_to_carrot: 1.0  # seconds
\end{lstlisting}

The controller will abort any path that leads through a lethal or near-lethal costmap cell within 1.0 second of projected travel time. This is a critical safety feature for a real vehicle.

\subsection{Resolution}

To test navigation in the current environment:
\begin{enumerate}[nosep]
    \item Choose a goal in a direction clear of obstacles (use costmap data to identify free directions)
    \item Move the vehicle to an open area before setting goals
    \item Or temporarily increase the collision time threshold for bench testing (not recommended for real driving)
\end{enumerate}

%% ============================================================
\section{CycloneDDS Multi-Interface Failure on Humble}
\label{sec:cyclonedds}

\subsection{Symptom}

An attempt to configure CycloneDDS to bind to both the local Ethernet interface (\texttt{eno1}) and the Tailscale VPN interface (\texttt{tailscale0}) for remote RViz visualization caused all ROS2 nodes to crash on startup:

\begin{lstlisting}[style=error]
[rmw_cyclonedds_cpp]: rmw_create_node: failed to create domain,
  error Error
Failed to find a free participant index for domain 0
\end{lstlisting}

\subsection{Root Cause}

CycloneDDS 0.10.5 (the version shipped with ROS2 Humble) does not support multiple \texttt{<NetworkInterface>} elements in the XML configuration. The multiple-interface feature was added in later CycloneDDS versions. When the configuration contained two \texttt{<NetworkInterface>} entries:

\begin{lstlisting}
<Interfaces>
  <NetworkInterface name="eno1"/>
  <NetworkInterface name="tailscale0"/>
</Interfaces>
\end{lstlisting}

the DDS domain initialization failed entirely, preventing any ROS2 node from starting.

Alternative approaches (\texttt{NetworkInterfaceAddress}, binding to \texttt{0.0.0.0}) also failed for the same underlying reason.

\subsection{Fix}

Reverted to the original single-interface CycloneDDS configuration:

\begin{lstlisting}
<Interfaces>
  <NetworkInterface autodetermine="true"/>
</Interfaces>
\end{lstlisting}

\subsection{Workaround for Remote Visualization}

Since cross-machine DDS over Tailscale is not feasible with CycloneDDS on Humble, the following alternatives are available:

\begin{enumerate}[nosep]
    \item \textbf{SSH X11 forwarding}: Run RViz on the Jetson, display on the laptop via \texttt{ssh -Y}. Slow for 3D rendering but functional.
    \item \textbf{Foxglove Studio}: Web-based visualization via \texttt{rosbridge\_server}. No DDS cross-machine configuration needed.
    \item \textbf{Upgrade to Jazzy/Rolling}: Newer CycloneDDS versions (0.11+) support multiple network interfaces natively.
\end{enumerate}

%% ============================================================
\section{Summary of Configuration Changes}

\begin{table}[h]
\centering
\small
\begin{tabular}{@{}llll@{}}
    \toprule
    File & Parameter & Before & After \\
    \midrule
    \texttt{xsens.yaml} & \texttt{pub\_odometry} & (absent, default \texttt{true}) & \texttt{false} \\
    \texttt{cyclonedds.xml} & Multi-interface & Attempted & Reverted to autodetermine \\
    \bottomrule
\end{tabular}
\caption{Configuration changes during navigation testing}
\label{tab:summary}
\end{table}

%% ============================================================
\section{Final Test Results}

After all fixes, the navigation stack operates correctly:

\begin{table}[h]
\centering
\small
\begin{tabular}{@{}lll@{}}
    \toprule
    Test & Result & Notes \\
    \midrule
    TF tree connectivity & Pass & \texttt{map $\to$ odom $\to$ base\_link}, no loops \\
    LiDAR costmap integration & Pass & 4100/10000 local cells marked \\
    Global costmap & Pass & 44k obstacle, 68k free, 138k unknown \\
    SmacPlannerHybrid path planning & Pass & 49 waypoints, 30\,ms planning time \\
    Direct \texttt{compute\_path\_to\_pose} & Pass & Dubin curves respecting 2.31\,m radius \\
    \texttt{navigate\_to\_pose} goal acceptance & Pass & Goal accepted by BT navigator \\
    RPP collision detection & Pass & Correctly aborted path with obstacle ahead \\
    Nav2 lifecycle activation & Pass & All 6 managed nodes active \\
    Velodyne point cloud & Pass & $\sim$20\,Hz continuous \\
    Xsens IMU & Pass & 100\,Hz, no TF conflicts \\
    Dual EKF & Pass & Both filters running, \texttt{map} frame published \\
    Vehicle mode switching & Pass & N $\to$ D via ActuatorCommand \\
    \bottomrule
\end{tabular}
\caption{Navigation stack test results}
\label{tab:test-results}
\end{table}

The only remaining item is to test with the vehicle in an open area where the goal direction is clear of obstacles, allowing the Regulated Pure Pursuit controller to execute the full path without triggering collision detection.

\end{document}
